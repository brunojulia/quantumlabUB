%%%%%%%%%%%%%%%%%%%%%%%%%%%%%%%%%%
%%%%%%%%%%%%%%%%%%%%%%%%%%%%%%%%%%
%Plantilla pel Treball Fi de Grau
%%%%%%%%%%%%%%%%%%%%%%%%%%%%%%%%%%
%%%%%%%%%%%%%%%%%%%%%%%%%%%%%%%%%%%%%%%%%%%
\documentclass[singlecolumn]{revtex4}
%%%%%%%%%%%%%%%%%%%%%%%%%%%%%%%%%%%%%%%%%%%
\usepackage{graphicx,epsfig}
\usepackage{amsmath}
\usepackage{amsfonts}

\usepackage{fancyhdr}

%%%%%%%%%%%%%%%%%%%%%%%%%%%%%%%%%%%%%%%%%%%
%%%%%%%%%%%%%%%%%%%%%%%%%%%%%%%%%%%%%%%%%%%

\def\beq{\begin{equation}}{\it}
\def\eeq{\end{equation}}
\def\beqa{\begin{eqnarray}}{\it}
\def\eeqa{\end{eqnarray}}

\begin{document}


%%%%%%%%%%%%%%%%%%%%%%%%%%%%%%%%%%%%%%

%Això és un comentari que no surt al text.

%%%%%%%%%%%%%%%%%%%%%%%%%%%%%%%%%%%%%%
%Es pot ometre
%\pagestyle{fancy}
%\lhead{\bf TITLE}
%\rhead{Student's name}
%\lfoot{Treball de Fi de Grau}
%\rfoot{Barcelona, June 2013}
%%%%%%%%%%%%%%%%%%%%%%%%%%%%%%%%%%%%%%


\title{Notes on the solution to 1D Schrödinger equation for piecewise 
potentials}
\author{bjd}
\affiliation{Facultat de F\'{\i}sica, Universitat de Barcelona, Diagonal
645, 08028 Barcelona, Spain.}
%\date{\today}

\begin{abstract}
Brief rendition of the way to solve the Schrödinger equation for piecewise potentials
\end{abstract}

\maketitle

%\tableofcontents

%%%%%%%%%%%%%%%%%%%%%%%%%%%%%%%%%%%%%%%%%%%%%%%%%%%%%%%%%%%%%%%%%%%%%%%%%%%

We consider the time independent 1D Schrödinger equation, (in natural units)
\beq
-{1\over 2} \partial^2_x \psi(x) + V(x) \psi(x) =  E\psi(x)\,.
\eeq
Where the potential is taken to be piecewise constant, that is, 
\beq
V(x)= V_k \; x_k < x < x_{k+1}\,.
\eeq
where $\{x_k\}$ is a partition of the interval $[0,L]$, and $k=0,\dots,N$.  
For simplicity we take the utmost left and right potentials to be infinite, 
that is, the wave function is zero at $x=0$ and $x=L$. For each interval, $[x_k,x_{k+1}]$ the 
general solution of the Schrödinger equation reads, 
\beq
\psi_k(x) = A_k e^{i\kappa_k x} + B_k e^{-i \kappa_k x}
\eeq
where $\kappa_k=\sqrt{2 (E-V_k)}$. 

First we consider the first and last intervals,
\beqa
\psi_0(x) &=& A_0 e^{i\kappa_0 x} + B_0 e^{-i \kappa_0 x}\nonumber\\ 
\psi_N(x) &=& A_N e^{i\kappa_N x} + B_N e^{-i \kappa_N x} 
\eeqa
the boundary conditions imply,
\beqa
\psi_0(x) &=& A_0\left( e^{i\kappa_0 x} - e^{-i \kappa_0 x}\right) = A_0 \,2\, i \, \sin(\kappa_0 x) \nonumber\\ 
\psi_N(x) &=& A_N e^{i \kappa_N L} \left( e^{i\kappa_N (x-L)} - e^{-i \kappa_N (x-L)} \right)= 
A_N \,2\, i\, \sin(\kappa_N (x-L))
\label{firstlast}
\eeqa
  
If we consider two inner intervals we have to impose continuity of the wave function 
and of its first derivative, that reads, 
\beqa
\psi_k(x_{k+1})&=&\psi_{k+1}(x_{k+1}) \nonumber\\
\psi'_k(x_{k+1})&=&\psi'_{k+1}(x_{k+1}) 
\eeqa
which read,
\beqa
A_k e^{i\kappa_k x_{k+1}} + B_k e^{-i\kappa_k x_{k+1}} &=&  A_{k+1} e^{i\kappa_{k+1} x_{k+1}} + B_k e^{-i\kappa_{k+1}x_{k+1}} 
\nonumber\\
  A_k i\kappa_k e^{i\kappa_k x_{k+1}} - B_k i \kappa_k e^{-i\kappa_k x_{k+1}} &=&  
A_{k+1} i \kappa_{k+1} e^{i\kappa_{k+1} x_{k+1}} - B_k i \kappa_{k+1} e^{-i\kappa_{k+1}x_{k+1}} \,.
\eeqa
These can be written in matrix form as,
\beq
\left(
\begin{matrix}
e^{i\kappa_k x_{k+1}} & e^{-i\kappa_k x_{k+1}} \cr
i\kappa_k e^{i\kappa_k x_{k+1}}  & -i \kappa_k e^{-i\kappa_k x_{k+1}} d\cr
\end{matrix}
\right)
\left(
\begin{matrix}
A_k\cr
B_k
\end{matrix}
\right)
=
\left(
\begin{matrix}
e^{i\kappa_{k+1} x_{k+1}} & e^{-i\kappa_{k+1} x_{k+1}} \cr
i\kappa_{k+1} e^{i\kappa_{k+1} x_{k+1}}  & -i \kappa_{k+1} e^{-i\kappa_{k+1} x_{k+1}} d\cr
\end{matrix}
\right)
\left(
\begin{matrix}
A_{k+1}\cr
B_{k+1}
\end{matrix}
\right)
\eeq
which can be written as,
\beq
{\cal M}(\kappa_k,x_{k+1}) \phi_k = {\cal M}(\kappa_{k+1},x_{k+1}) \phi_{k+1}
\eeq
where,
\beq
{\cal M}(\kappa,x) = \left(
\begin{matrix}
e^{i\kappa x} & e^{-i\kappa x} \cr
i\kappa e^{i\kappa x}  & -i \kappa e^{-i\kappa x} \cr
\end{matrix}
\right)
\qquad 
\phi_k=\left(
\begin{matrix}
A_{k}\cr
B_{k}
\end{matrix}
\right)
\eeq

This allows us to solve the wave function in the $k+1$ interval from the values in the 
$k$ interval,
\beq
\phi_{k+1} = 
{\cal M}^{-1}(\kappa_{k+1},x_{k+1}) 
{\cal M}(\kappa_{k},x_{k+1})
\phi_k
\eeq 
where the inverse matrix reads, 
\beq
{\cal M}^{-1}(\kappa,x) = \left(
\begin{matrix}
{1\over 2} e^{-i\kappa x} &{-i\over 2\kappa}  e^{-i\kappa x} \cr
{1\over 2} e^{i\kappa x}  & {i\over 2 \kappa} e^{i\kappa x} \cr
\end{matrix}
\right)
\eeq

\subsection{how to find the eigenenergies?}
To find the energy quantization we basicaly write the wave function in the last interval as a function of the wave function in the first interval, 
\beq
\hat{\phi}_N = 
{\cal M}^{-1}(\kappa_{N},x_{N}) 
{\cal M}(\kappa_{N-1},x_{N})
\dots
{\cal M}^{-1}(\kappa_{j+1},x_{j+1}) 
{\cal M}(\kappa_{j},x_{j+1})
\dots
{\cal M}^{-1}(\kappa_{1},x_{1}) 
{\cal M}(\kappa_{0},x_{1})
\phi_0
\eeq
with $\phi_0=\{A_0,-A_0\}$. This should indeed be equal to $\phi_N$, that we wrote 
in Eq.~(\ref{firstlast}).

 Now for instance we do, 
\beq
\hat{\phi}_N(1)/\hat{\phi}_N(2) - \phi_N(1)/\phi_N(2)=0
\eeq
and look for values of $E$ that solve the above equation. Notice in the above 
equation the values of $A_0$ and $A_N$.

\begin{thebibliography}{99}

%\bibitem{latex}Helmut Kopka and Patrick W. Daly, \textsl{A Guide to
%\LaTeX: Document Preparation for Beginners and Advanced Users}, 3rd. ed. (Addison-Wesley, Reading, MA, 1999).
\end{thebibliography}





\end{document}
